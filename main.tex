\documentclass{article}
\usepackage[a4paper, portrait, margin=0.5in]{geometry}

\title{Cloud Chamber}
\author{Absolute Physics }
\date{{\large November 2019}}

\begin{document}
\maketitle
\large{ A Wilson Cloud Chamber is a particle detecting system used to visualize the path of ionizing radiation. It comprises of a sealed environment containing a supersaturated vapor of alcohol. The interaction between the vapor and ionizing radiation forms streaks in the cloud chamber which help to visualize the passage of ionizing radiation.}
\section*{Aim}
\large{Build a cloud chamber to visualize the path of ionizing radiation}
\section*{Resources}
\large{\begin{enumerate}
  \item Glass tanks/containers
  \item Hot bag
  \item Black felt
  \item Dry ice
  \item Metal Sheet
  \item 99\% \textgreater propanol/ethanol
  \item Light Source
  \item Radioactive Sources
\end{enumerate}}

\begin{center}
\section*{Experiment 1}
\end{center}
\large{\begin{enumerate}
    \item Attach black felt to underside of glass tank by using plasticine as supports on sides of the tank
    \item Spray ethanol/propanol on felt
    \item Crush dry ice and compact it between 2 metal trays
    \item Place glass tank upside-down on the metal tray, which should be cold due to the dry ice
    \item Shine a light through and observe the cloud
\end{enumerate}}

\section*{Observations}
\large{\begin{enumerate}
    \item No Cloud observed
    \begin{itemize}
      \item Problems
      \begin{itemize}
          \item Rate of evaporation
          \item Contact of felt with top surface of glass tank
          \item Felt keeps falling
      \end{itemize}
    \end{itemize}
    \item Ethanol present in liquid form on metal tray obscuring observations
    \item Lustrous surface of metal tray makes it difficult to observe the presence of a cloud
\end{enumerate}}

\section*{Improvements made based on observations from Experiment 1}
\large{\begin{enumerate}
  \item No Cloud observed
  \begin{itemize}
      \item Problems
      \begin{itemize}
          \item Hot bag placed on top of the glass tank to increase the rate of evaporation
          \item Use of glue to achieve contact with surface
          \item Use of glue to permanently attach felt to the tank surface
      \end{itemize}
      \end{itemize}
    \item Continually wipe the tray and glass to prevent condensation from obscuring observations
    \item Painted metal tray matte black to better observe the cloud
\end{enumerate}}

\begin{center}
\section*{Experiment 2}
\end{center}
\large{\begin{enumerate}
    \item Black felt attached to underside of glass tank by using glue.
    \item Spray ethanol/propanol on felt
    \item Crush dry ice and compact it between 2 metal trays
    \item Place glass tank upside-down on the metal tray, which should be cold due to the dry ice
    \item Place hot bag on top of the tank to increase the rate of evaporation of ethanol.
    \item Shine a light through and observe the cloud
    \item Before repeating experiment clean surface of metal tray
\end{enumerate}}

\section*{Observations}
\large{\begin{enumerate}
  \item Cloud Observed
  \item Faint streaks observed
\end{enumerate}}



\end{document}
